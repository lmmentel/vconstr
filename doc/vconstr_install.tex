\documentclass[10pt]{article}
\usepackage{multicol}
\usepackage[top=2cm, bottom=2cm, left=2cm, right=2cm]{geometry}
\usepackage{enumerate}
\usepackage{calc}
% rules for the tables
\usepackage{booktabs}
% minted is needed to generated sytax highlighted code listings
\usepackage{minted}
%headers and footers
\usepackage{fancyhdr}
% hyperref and link color definitions
\usepackage{hyperref}
\hypersetup{
            pdfborder={0,0,0},
            colorlinks=true,
            citecolor=blue,
            linkcolor=red,
            urlcolor=blue,
            }
% required to get the number of pages
\usepackage{lastpage}
% required to generate index
%\usepackage{makeidx}
%\makeindex                        % if commented index will not be generated 
%support for directory tree
\usepackage{dirtree}
\usepackage{xcolor}
\usepackage{cite}
% specify header and footer
\pagestyle{fancy}
\lhead{}
\chead{}
\rhead{}
\lfoot{\href{mailto:lmmentel@gmail.com}{lmmentel@gmail.com}}
\cfoot{}
\rfoot{\thepage\ / \pageref{LastPage}}
\renewcommand{\headrulewidth}{0.4pt}
\renewcommand{\footrulewidth}{0.4pt}
\fancypagestyle{first}{%
    \renewcommand{\headrulewidth}{0pt}%
    \lhead{}
    \chead{}
    \rhead{}
    \lfoot{\href{mailto:lmmentel@gmail.com}{lmmentel@gmail.com}}
    \cfoot{}
    \rfoot{\thepage\ / \pageref{LastPage}}
      }
% needed to have the same foot and head on the index page
\fancypagestyle{plain}{%
    \renewcommand{\headrulewidth}{0.4pt}
    \renewcommand{\footrulewidth}{0.4pt}
    \lhead{}
    \chead{}
    \rhead{}
    \lfoot{\href{mailto:lmmentel@gmail.com}{lmmentel@gmail.com}}
    \cfoot{}
    \rfoot{\thepage\ / \pageref{LastPage}}
      }

% Don't print section numbers
%\setcounter{secnumdepth}{0}

% -----------------------------------------------------------------------

\begin{document}
\thispagestyle{first}
\footnotesize

\begin{center}
    \Large{\textbf{Remarks on Vconstr code interfaced with GAMESS-US}} \\ 
    \normalsize{\textbf{basic information, compilation and use}} \\
    \normalsize{Last modified: \today}
\end{center}

\section{General Information}
\begin{minted}[mathescape,
               linenos,
               numbersep=2pt,
               gobble=0,
               frame=lines,
               framesep=2mm]{bash}
echo Files used on the master node master were:
ls -lF 
\end{minted}
\section{dbrain.f requires the following data}
    \begin{itemize}
        \item \verb!hmatx! core hamiltonian matrix 
        \item \verb!tsmat! kinetic energy operator integrals
        \item \verb!vnmat! electron-nuclear attraction operator integrals
        \item \verb!vmopao! molecular orbitals in primitive ao basis
        \item \verb!vmoao! moelcular orbitals in mo basis
        \item \verb!pmo! mo density matrix in mo basis
        \item \verb!pnomo! ci density matrix in mo basis  
    \end{itemize}
\section{dsfun.x input}        
    \begin{table}[H]
        \begin{center}
        \caption{\label{tab:dsfun-input}dsfun input parameters}
        \begin{tabular*}{0.6\textwidth}{lll}
        \toprule
        \multicolumn{1}{c}{name} & \multicolumn{1}{c}{type} & \multicolumn{1}{c}{comment} \\ \midrule
        \verb!title! & character(len=8) & \\
        \verb!nmos!  & integer & number of molecular orbitals \\
        \verb!occ!   & real(1:nmos) & occupation numbers \\
        \verb!df!    & real & shift \\
        \verb!nppr!  & integer & pprint, intermediate storage of Vxc at every iteration \\
        \verb!scfdmp! & real & damping \\
        \verb!lrfun! & logical & linear response \\
        \verb!dvdmp! & real & linear response procedure damping \\
        \verb!lfield! & logical & electric field \\
        \verb!fzyx! & real(1:3) & components of the electric field \\
        \verb!nvpr! & integer & vprint \\
        \verb!lsym! & logical & symmetrize the orbitals \\
        \verb!lintsm! & logical & determine symmetry using orbitals(if false h-matrix) \\
        \verb!ismo! & integer & mo section in the atmol dumpfile \\
        \verb!isno! & integer & no section in the atmol dumpfile \\
        \verb!isao! & integer & adapt section in the atmol dumpfile from which the Vmopao matrix is restored \\
        \bottomrule
        \end{tabular*}
        \end{center}
    \end{table}
%\printindex
\section{iPrint}
\begin{enumerate}[1 - ]
    \setcounter{enumi}{-1}
    \item minimal printing, standard dsfun printing (this option is the default),
    \item print one electron integrals (kinetic and nuclear-electron attraction) over AO's and nuclar repulsion energy 
    read from Gamess-US dictionary file,
    \item same as \verb!iPrint=1! plus orbitals (HF MO's in AO, NO's in AO), NO occupation numbers - 
    read from Gamess-US dictionary file and calculates in the program (NO's in MO) and densities 
    (CI-NO density in MO, HF-MO density in MO),
    \item same as \verb!iPrint=2! plus basis set information as read from the Gamess-US basinfo file,
    \item same as \verb!iPrint=3! plus Hartree potential at every iteration of KS potential,
    \item same as \verb!iPrint=4! plus values, gradients, and laplacians for every AO and grid point,
    {\color{red}Warning!} for large numbers of AO and grid points this will create very big files!
\end{enumerate}


\bibliographystyle{lmentelbst_sc}
\bibliography{library}
\end{document}
